\chapter{帰納的関数論(Recursive function theory)}
帰納的関数は,
\[
 \N ^n \rightarrow \N
\]
の写像であり,この章で述べていくような特徴がある.これは,次章の「計算可
能性」を議論するための道具として用いられる.

\section{Peanoの公理}
Peanoの公理は,自然数を定義する公理である.
\begin{itemize}
 \item 0は自然数
 \item 自然数の次は自然数
 \item 以上で定義されるものだけが自然数
\end{itemize}
このように自然数が定義される.現代的には,自然数とはPeanoの公理を満たす
ものとされるらしい.

\section{原始帰納的関数(Primitive recursive function)}
\mystrong{原始帰納的関数}は,以下のいずれかで定義される.
\begin{description}
 \item[ゼロ関数 (zero function)] \mbox{} \\
            \[
             zero(x) = 0
            \]
            Peano公理から0の存在は言えるので,このような関数は存在する.

 \item[後者関数 (successor function)] \mbox{} \\
            \[
             succ(x) = x + 1
            \]
            Peanoの公理の1つ,「自然数の次は自然数」を反映した関数.

 \item[射影関数 (projection function)] \mbox{} \\
            \[
             proj_k (x_1, \cdots , x_k , \cdots , x_n) = x_k
            \hspace{2ex} (k \in \{1, \cdots , n\})
            \]
            これはPeanoの公理とは関係なく,「引数の中から1つを選ぶ関数」
            を定義しただけ.

 \item[関数合成] \mbox{} \\
            \[
            h(x_1 , \cdots , x_n) = g ( f_1 (x_1, \cdots , x_n), \cdots
            , f_n (x_1 , \cdots, x_n))
            \]
            ただし,$g, f_k$は原始帰納的関数.

 \item[原始帰納法 (primitive recursion)] \mbox{} \\
            \begin{eqnarray*}
            &h& ( x_1 \cdots , x_n , 0) = f ( x_1 , \cdots , x_n) \\
             &h& (x_1, \cdots , x_n , y+1) = g(x_1, \cdots, x_n, y,
              h(x_1 , \cdots, x_n , y))  \footnotemark
            \end{eqnarray*}
            ただし,$f, g$は原始帰納的関数.$h$の再帰呼出しは,最後の引
            数が必ず小さくなるため,最後の引数が必ず$0$になり,それは1行
            目の定義から計算できる.計算は有限ステップ.
\end{description}
\footnotetext{ここで$y+1$のような加算が出てくるが,加算は\ref{sec:06原始
関数の例}で定義される.}

\section{原始帰納的関数の例} \label{sec:06原始関数の例}
\begin{description}
 \item[定数関数] \mbox{} \\
            任意の定数関数は,有限回の$succ$と$0$の合成で可能.
            \begin{myexample}{'2'の定義}
             \begin{eqnarray*}
              two(x) &=& succ(one(x)) \\
              one(x) &=& succ(zero(x))
             \end{eqnarray*}
            \end{myexample}
            以降,定数が突然出てきても,それは定数関数で定義されたものと
            思いましょう.

 \item[引数の順序の変更] \mbox{} \\
            \begin{myexample}{2引数の順序の変更}
             \begin{eqnarray*}
              f(x, y) &=& g(proj_2(x, y)), proj_1(x, y)) \\
              &=& g(y, x)
             \end{eqnarray*}
            \end{myexample}

 \item[加算] \mbox{} \\
            \begin{eqnarray*}
             &add&(x, 0) = proj_1(x) = x \\
             &add&(x, y+1) = succ (add(x, y)) = \cdots = x + (y + 1) = x
              + y ) + 1
            \end{eqnarray*}

 \item[乗算] \mbox{} \\
            \begin{eqnarray*}
             &mult&(x, 0) = 0 \\
             &mult&(x, y+1) = add(x, mult(x, y)) = \cdots = x \times
              (y+1) = x \times y + x
            \end{eqnarray*}

 \item[前者関数(predcessor)] \mbox{} \\
            \begin{eqnarray*}
             pred(0) &=& 0 \\
             pred(x+1) &=& proj_1 (x, pred(x)) \hspace{3ex} (原始帰納法
              と射影関数を用いた) \\ 
             &=& x
            \end{eqnarray*}
 \item[減算] \mbox{} \\
            ここでは,自然数の枠組みに収まるように減算を定義する.
            \begin{eqnarray*}
             &sub&(x, 0) = x \\
             &sub&(x, y+1) = pred(sub(x, y)) = \cdots = x - (y+1) = (x-y)-1
            \end{eqnarray*}
\end{description}
通常用いる関数のほとんどは原始帰納的である.

\section{原始帰納的でない関数}
原始帰納的でない関数の例として,Ackermann関数が有名.
\begin{eqnarray*}
 &Ack&(0, y) = y + 1 \\
 &Ack&(x+1, 0) = Ack(x, 1) \\
 &Ack&(x + 1, y + 1) = Ack(x, Ack(x+1, y))  \footnotemark
\end{eqnarray*}
\footnotetext{このような関数は\mystrong{多重帰納法}という.これは原始帰
納関数の枠組みにはない方法.}
定義の適用のたびに$x$か$y$の少なくとも一方は小さくなるので,この計算は有限ステップで終わる.

しかし,これと同じことをする関数は,原始帰納関数の枠組みで作ることができ
ない.

\begin{table}
 \caption{Ackermann関数を具体的に見た表}
 \begin{center}
  \begin{tabular}{c|c|c|c|c|c|c|c}
   \hline
   $x$ \ $y$ & 0 & 1 & 2 & 3&4&5&6 \\
    \hline \hline
   0&1&2&3&4&5&6&7\\ \hline
   1&2&3&4&5&6&7&8\\ \hline
   2&3&5&7&9&11&13&15\\ \hline
   3&5&13&29&61&125&253&509\\ \hline
   4&13&65533&$\cdots$&$\cdots$&$\cdots$&$\cdots$&\\ \hline
  \end{tabular}
  \label{tb:06Ackermann関数を具体的に見た表}
 \end{center}
\end{table}
表\ref{tb:06Ackermann関数を具体的に見た表}から見て取れるように,Ackermann関数の値は,第1引数の増加に伴って急速に大きくなる.実は,原始帰
納関数には以下のような定理が成り立つ.\footnote{これの証明は扱ってません.}
\begin{mytheorem}
どんな原始的帰納関数$g$についても,ある定数$c$に対して,
\[
 g(x_1, \cdots , x_n) < Ack(c, \max (x_1, \cdots , x_n))
\]
が成り立つ.
\end{mytheorem}
\therefore Ackermann関数は原始帰納的に定義できない.

\section{原始帰納的述語}
真偽値を返す原始的関数を,\mystrong{原始帰納的述語}という.

\section{帰納的関数(Recursive function)}
原始帰納的関数の定義形式に加えて,\mystrong{最小解関数}を許す.最小解関
数とは,原始帰納的述語$p(n, x_1, \cdots , x_k)$が真となる最小の$n$を値と
する関数.

\mystrong{帰納的関数は,TMと同等の計算能力を持つ.}

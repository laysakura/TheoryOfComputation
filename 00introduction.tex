%#!platex main.tex

\chapter{はじめに}

\section{この資料について}
東京大学工学部電子情報工学科の3年冬学期に行われた授業,「計算論」の講義
ノートです.

\section{資料の方針}
\begin{itemize}
 \item 授業でやった内容はカバーします.
 \item 授業でやってないことは基本的にやりません.
 \item ただし,授業では扱ってないが,理解が深まりそうな例とかは扱います.
 \item 重要そうなワードは\mystrong{赤字}にしてます.赤シート使って勉強す
       れば単位が来るかは知りません.
\end{itemize}

\section{作った動機}
最も大きな動機は,\LaTeX の練習です.色々なスタイルやマクロを試してみた
かったので.ソースも公開していますので,そちらもよろしければご参照くださ
い.\\
\url{https://github.com/laysakura/TheoryOfComputation}

あと,個人的に内容にとても興味が持てました.特にコンピュータを触る人にとって
は面白い話だと思います.だからこの講義で\LaTeX の練習をすることにしたわ
けです.

\section{お願い}
この資料も不完全な部分が多いかと思います.もしもご意見などくださるのであ
れば,\url{lay.sakura@gmail.com} (中谷)までよろしくお願いします.

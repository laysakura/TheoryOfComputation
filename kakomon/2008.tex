\documentclass[report,a4j]{jsbook}

% === Packages ===
\usepackage{amsmath}
\usepackage{txfonts}
\usepackage{ascmac}
\usepackage{fancybox}

% necessary for 'leftbar' env.
\usepackage{framed}

% necessary for 'breakrectbox' env.
\usepackage{emathPbb}
\usepackage{emathPb}

% necessary for 'breakitembox' env.
\usepackage{eclbkbox}
\usepackage{itembbox}
\usepackage{itembkbx}
\usepackage{jquote}
\usepackage{emathC}

% necessary for tree env.
% Usage:
% See: ~/lecture/jikken/theory_of_computation/03context_free_grammar.tex
\usepackage{epic,eepic,eclarith,treeprint}

% Used to use URL in bibtex
\usepackage{url}

% Used to display jpg,png image
% Used like:
% \myfigure{hoge.jpg}{this is a caption}
\usepackage[dvipdfmx]{graphicx}

% Used to display jpg images
% without '$ ebb *.jpg'
% Usage:
% \includegraphics[autoebb]{hoge.jpg}
% NOTE: this cannot be used for png,gif
\usepackage{mediabb}

% Uses color
% Color samples are in
% /usr/local/teTeX/share/texmf-dist/doc/latex/xcolor/xcolor.dvi
\usepackage[svgnames,dvipdfm]{xcolor}

% Bueatiful source code style.
% Used like:
% \lstinputlisting[label=src:Point.h, caption=Point.h]{../Point.h}
\usepackage{listings,jlisting}

% Self-defined macros.
% See: /usr/local/teTeX/share/texmf/ptex/platex/mymacro/mymacro.sty
\usepackage{mymacro}


% === Document Class Setings ===
\renewcommand{\figurename}{図}
\renewcommand{\tablename}{表}
\renewcommand{\refname}{参考}
\renewcommand{\bibname}{参考}

\begin{document}


2008年「計算論」期末試験の解答です.合ってるかは知りません.

\section*{【問1】以下の各問に...}
\subsection*{解答}

\subsubsection*{(1)}
\mystrong{
\[
 A \cap B = \{x | \,\, x \in A \, かつ \, x \in B \}
\]
すなわち,$A$にも$B$にも含まれるアルファベットから構成される言語.
}

\subsubsection{(2)}
\mystrong{
現在の状態,入力記号,スタックポインタの指す値.
}

\subsubsection{(3)}
\mystrong{
決定的プッシュダウンオートマトン(DPDA)は,非決定的有限オートマトン(NFA)
よりも,受理言語の範囲は広い.その理由を,
\begin{enumerate}
 \item DPDAに受理できてNFAに受理できない言語の例を示す
 \item DPDAでNFAの動作がエミュレートできることを示す
\end{enumerate}
ことにより,説明する.}

\mystrong{
まず,DPDAは無限に大きなスタックを持
つため,無限の長さの言語を受理する能力がある.一方,NFAは状態数が有限で
あるため,その能力はない.
}

\mystrong{
また,次のようにして,DPDAでNFAをエミュレートできる.まず,DPDAの有限制
御部の状態として,エミュレートしたいNFAの状態をコピーする.NFAでは,同じ
入力に対して異なる状
態間の遷移が起こることがあるが,これをいかにしてDPDAでエミュレートするか
を説明するため,次のような簡単な例を考える.いま,現状態が$S_0$であったと
する.このとき,入力$x$を受け取ると,NFAでは$S_1, S_2$に状態遷移できるとし
よう.これをDPDAでエミュレートするには,スタックに「$S_1$へ遷移するつもり」という
情報を記憶しておいて,$S_1$に移動する.もし$S_1$へ遷移したせいで,言語を受
理できるpathをたどれなかった場合,遷移を全く逆にたどり,現状態$S_0$まで
戻ってくる.このときスタックポインタを見ると,「$S_1$に移動するつもり」
という情報が残っているので,次は$S_2$をたどれば良い.この動作を再帰的に
繰り返していくことで,DPDAでNFAをエミュレートすることができる.
}

(説明下手すぎワロエナイ)

\section*{【問2】下記の一群の...}
\subsection*{解答}
\mystrong{
\begin{eqnarray*}
 &F& \rightarrow 0 \\
 &F& \rightarrow 1 \\
 &Op& \rightarrow + \\
 &Op& \rightarrow - \\
 &S& \rightarrow 0 \\
 &S& \rightarrow 1 \\
 &S& \rightarrow 0 \,\, Op \,\, S \\
 &S& \rightarrow 1 \,\, Op \,\, S
\end{eqnarray*}
}

\subsection*{ポイント}
\begin{itemize}
 \item 解答の順番で考えると少し楽な気がしないでもないです
\end{itemize}

\section*{【問4】アルファベットが...}
\subsection*{解答}
\myfigure[width=150truemm]{4.eps}{問4の解答}

\mystrong{初期状態はS0,最終状態は,S0-S6.}

\subsection*{ポイント}
\begin{itemize}
 \item 各状態から,すべてのアルファベット(a, b, c)の矢印が出ていることを
       チェック
 \item 問題文に「それ以外には制約がない」とわざわざ書いているので,与え
       られた3条件のうちいずれかを満たさない様な言語だけが受理されないこ
       とに注意.空列$\epsilon \hspace{1ex}$(S0で最終状態を迎える)も受理されることに
       も注意.
 \item 答案にはもう少しわかり易く矢印描く方が良いでしょうね
\end{itemize}

\nocite{*}                      % display materials not quoted
\bibliographystyle{jplain}
\bibliography{report}

\end{document}



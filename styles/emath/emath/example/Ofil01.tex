\begin{edaenumerate}<5>'(\Ofil{2}{\getcurrentenum})'
  \item aa
  \item bb
  \item cc
  \item dd
  \item ee
  \item ff
  \item gg
  \item hh
  \item ii
  \item jj
\end{edaenumerate}
